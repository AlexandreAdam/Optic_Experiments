\documentclass[10pt,letterpaper,twocolumn]{article}

%2012-10-01 - Document préparé par David Lafrenière, pour le cours PHY3040.

%Pour langue et caractères spéciaux
\usepackage[french]{babel} 
\usepackage[T1]{fontenc}
\usepackage{lmodern}
\usepackage[utf8]{inputenc}

\usepackage[backend=biber, style=numeric]{biblatex}
\addbibresource{reference.bib}
  
%Package for math expression
\usepackage{amsmath}
\usepackage{amsthm,amstext,amsfonts,bm,amssymb,amsthm}
\usepackage{mathtools}
\usepackage{bm}
\usepackage{gensymb}
\usepackage{mathrsfs}
\usepackage{physics}

%Package for drawings
\usepackage{tikz}
\usetikzlibrary{calc,patterns,angles,quotes}
\usepackage[compat=1.1.0]{tikz-feynman}
%\usetikzlibrary{3d}
\usetikzlibrary{decorations.pathreplacing}
%\usepackage{lineno}

%Package pour les symbole astronomiques
%\usepackage{wasysym}

\usepackage{hyperref}

%Pour ajuster les marges
\usepackage[top=2cm, bottom=2cm, left=2cm, right=2cm, columnsep=20pt]{geometry}

% Pour la commande onecolabstract (résumé 1 pleine largeur)
\usepackage{abstract}
	\renewcommand{\abstractnamefont}{\normalfont\bfseries}
	\renewcommand{\abstracttextfont}{\normalfont\itshape}

% Pour les titres de section/sous-section
\usepackage[compact]{titlesec}
\titleformat{\section}{\large\bfseries}{\thesection}{1em}{}
\titleformat{\subsection}{\normalsize\bfseries}{\thesubsection}{1em}{}
\titleformat{\subsubsection}{\normalsize}{\thesubsubsection}{1em}{}

%Package for graphic expression
\usepackage{graphicx}
\usepackage{wrapfig}
\usepackage{float}
\usepackage{caption}
\usepackage{subcaption}
\usepackage{enumitem}

%Shorthand for space and some math expressions
\newcommand{\s}{\hspace{0.1cm}}
\renewcommand{\Im}{\operatorname{\mathbb{I}m}}
\renewcommand{\Re}{\operatorname{\mathbb{R}}}
%Shorthand for partial differential
\newcommand{\partialD}[2]{\frac{\partial #1}{\partial #2}}
%Shorthand for \left(\right)
\DeclarePairedDelimiter\autobracket{(}{)}
\newcommand{\br}[1]{\autobracket*{#1}}

\newcommand{\pyoutput}[2]{#2} % Simply output #2, use #1 as tag for python reader

%Pour inclure des adresse web
\usepackage{url}

%Titre
\title{\vspace{-10mm}\Large
Diffusion Raman %%%***éditer cette ligne***
\vspace{-4mm}}

%Auteur
\author{\large
Alexandre Adam
}
\date{\vspace{-8mm}}

\captionsetup{labelfont=bf, format=plain, font=footnotesize}

\begin{document}

\twocolumn[
\maketitle
\begin{onecolabstract} % 10 points

\vspace{4mm} %
\end{onecolabstract}
]

\section{Introduction}\label{intro} % 5 points


\section{Théorie}\label{sec:theorie} % 10 points
La diffusion Raman spontanée est un processus de troisième ordre (3 vertex) dans la théorie des perturbations quantiques. La figure \ref{fig:FeynmanD} montre le diagramme de Feynman représentant une diffusion Raman. Le phonon peut être absorbé par l'électron, dans quel cas on parle de du champ \textit{anti-Stokes}; il peut aussi être émis par l'électron (champ \textit{Stokes}). Cette symétrie se reflète dans le spectre de Raman, où la fréquence de résonance des modes du semi-conducteur se situe à égale distance de la fréquence incidente $\omega_i$. \par
Comme la paire électron-trou est virtuelle, \cite{Yu2010}

\begin{figure}[H]
	\centering
	\begin{tikzpicture}
		\tikzfeynmanset{
				every vertex={black, dot}, small
			}
		\begin{feynman}
			\vertex (a) {$\hbar \omega_i$};
			\vertex [right=of a] (b);
			\vertex [above right=of b] (c);
			\vertex [below right=of c] (d);
			\vertex [right=of d] (e) {$\hbar \omega_s$};
			\vertex [above=of c] (p) {$\hbar \omega_p$};
			
			\diagram* {
				(a) -- [photon] (b) -- [fermion, quarter left, edge label=$e^-$] (c) -- [fermion, quarter left, edge label=$e^-$] (d) -- [photon] (e);
				(b) -- [charged scalar, edge label'=$\text{trou}$, half right] (d);
				(c) -- [photon, edge label=$\text{phonon}$] (p);
			};
		\end{feynman}
		\node [below=of a] (t) {temps};
		\node [right=of t] (t1) {};
		\draw [->] (t) -- (t1); 
	\end{tikzpicture}
	\caption{Diagramme de Feynman dépictant le processus de diffusion Raman. Un photon (\leadsto) incident d'énergie $\hbar \omega_i$ interagit avec le milieu pour créer un paire électron-trou. Ensuite, un phonon optique est absorbé (anti-Stokes) ou émis (Stokes) par l'électron, qui se recombine avec le trou pour former un nouveau photon d'énergie $\hbar \omega_s$. }
	\label{fig:FeynmanD}
\end{figure}



\section{Méthodologie}\label{sec:metho} % 15 points


\section{Résultats et discussion}\label{sec:resultats} % 25 points


\section{Conclusion}\label{sec:conclusion} % 10 points

\section{Extra} % 15 points

\printbibliography
%\bibliographystyle{abbvr} 
%\bibliography{raman_scattering_bib}
%\begin{thebibliography}{1}
%\bibitem{ref1} texte de la référence
%\end{thebibliography}

\end{document}
