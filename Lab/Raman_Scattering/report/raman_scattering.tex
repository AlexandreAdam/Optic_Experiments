\documentclass[10pt,letterpaper,twocolumn]{article}

%2012-10-01 - Document préparé par David Lafrenière, pour le cours PHY3040.

%Pour langue et caractères spéciaux
\usepackage[french]{babel} 
\usepackage[T1]{fontenc}
\usepackage{lmodern}
\usepackage[utf8]{inputenc}

\usepackage[backend=biber, style=nature]{biblatex}
\addbibresource{reference.bib}
  
%Package for math expression
\usepackage{amsmath}
\usepackage{amsthm,amstext,amsfonts,bm,amssymb,amsthm}
\usepackage{mathtools}
\usepackage{bm}
\usepackage{gensymb}
\usepackage{mathrsfs}
\usepackage{physics}

%Package for drawings
\usepackage{tikz}
\usetikzlibrary{calc,patterns,angles,quotes}
\usepackage[compat=1.1.0]{tikz-feynman}
%\usetikzlibrary{3d}
\usetikzlibrary{decorations.pathreplacing}
\usetikzlibrary{automata,positioning}
%\usepackage{lineno}

%Package pour les symbole astronomiques
%\usepackage{wasysym}

\usepackage{hyperref}

%Pour ajuster les marges
\usepackage[top=2cm, bottom=2cm, left=2cm, right=2cm, columnsep=20pt]{geometry}

% Pour la commande onecolabstract (résumé 1 pleine largeur)
\usepackage{abstract}
	\renewcommand{\abstractnamefont}{\normalfont\bfseries}
	\renewcommand{\abstracttextfont}{\normalfont\itshape}

% Pour les titres de section/sous-section
\usepackage[compact]{titlesec}
\titleformat{\section}{\large\bfseries}{\thesection}{1em}{}
\titleformat{\subsection}{\normalsize\bfseries}{\thesubsection}{1em}{}
\titleformat{\subsubsection}{\normalsize}{\thesubsubsection}{1em}{}

%Package for graphic expression
\usepackage{graphicx}
\usepackage{wrapfig}
\usepackage{float}
\usepackage{caption}
\usepackage{subcaption}
\usepackage{enumitem}

%Shorthand for space and some math expressions
\newcommand{\s}{\hspace{0.1cm}}
\renewcommand{\Im}{\operatorname{\mathbb{I}m}}
\renewcommand{\Re}{\operatorname{\mathbb{R}}}
%Shorthand for partial differential
\newcommand{\partialD}[2]{\frac{\partial #1}{\partial #2}}
%Shorthand for \left(\right)
\DeclarePairedDelimiter\autobracket{(}{)}
\newcommand{\br}[1]{\autobracket*{#1}}

\newcommand{\pyoutput}[2]{#2} % Simply output #2, use #1 as tag for python reader

%Pour inclure des adresse web
\usepackage{url}

%Titre
\title{\vspace{-10mm}\Large
Diffusion Raman %%%***éditer cette ligne***
\vspace{-4mm}}

%Auteur
\author{\large
Alexandre Adam
}
\date{\vspace{-8mm}}

\captionsetup{labelfont=bf, format=plain, font=footnotesize}

\begin{document}

\twocolumn[
\maketitle
\begin{onecolabstract} % 10 points

\vspace{4mm} %
\end{onecolabstract}
]

\section{Introduction}\label{intro} % 5 points


\section{Théorie}\label{sec:theorie} % 10 points
La diffusion Raman spontanée est un processus de troisième ordre (3 vertex) dans la théorie des perturbations quantiques\supercite{Yu2010}. La figure \ref{fig:FeynmanD} montre le diagramme de Feynman représentant une diffusion Raman. Le phonon peut être absorbé par l'électron, dans quel cas on parle du champ \textit{anti-Stokes}; il peut aussi être émis par l'électron (champ \textit{Stokes}). \par

\begin{figure}[H]
	\centering
	\begin{tikzpicture}
		\tikzfeynmanset{
				every vertex={black, dot}, small
			}
		\begin{feynman}
			\vertex (a) {$\hbar \omega_i$};
			\vertex [right=of a] (b);
			\vertex [above right=of b] (c);
			\vertex [below right=of c] (d);
			\vertex [right=of d] (e) {$\hbar \omega_s$};
			\vertex [above=of c] (p) {$\hbar \omega_p$};
			
			\diagram* {
				(a) -- [photon] (b) -- [fermion, quarter left, edge label=$e^-$] (c) -- [fermion, quarter left, edge label=$e^-$] (d) -- [photon] (e);
				(b) -- [charged scalar, edge label'=$\text{trou}$, half right] (d);
				(c) -- [photon, edge label=$\text{phonon}$] (p);
			};
		\end{feynman}
		\node [below=of a] (t) {temps};
		\node [right=of t] (t1) {};
		\draw [->] (t) -- (t1); 
	\end{tikzpicture}
	\caption{Diagramme de Feynman dépictant le processus de diffusion Raman. Un photon (\leadsto) incident d'énergie $\hbar \omega_i$ interagit avec le milieu pour créer un paire électron-trou. Ensuite, un phonon optique est absorbé (anti-Stokes) ou émis (Stokes) par l'électron, qui se recombine avec le trou pour former un nouveau photon d'énergie $\hbar \omega_s = \hbar \omega_i \pm \hbar \omega_p$. }
	\label{fig:FeynmanD}
\end{figure}




\section{Méthodologie}\label{sec:metho} % 15 points

\subsection{Montage}
Pour obtenir les spectres Raman, on utilise le montage dépicté à la figure \ref{fig:montage}. Le faisceau laser est obtenu par l'excitation d'un plasma He-Ne avec une raie principale à $\lambda_L = 632.816\s nm$. Le plasma possède plusieurs raies secondaires dont la longueur d'onde se situe dans l'intervalle qui nous intéresse pour cette expérience ($640.9\s nm$ à $675.6\s nm$ pour le spectre Stokes en autre, qui correspond à un intervalle Raman $200\s cm^{-1}$ à $1000\s cm^{-1}$). Ces raies ont étés identifiés au début de l'expérience et prisent en compte dans l'analyse de données. \par

Les photons diffus par l'effet Raman sont émis dans une direction aléatoire à partir de l'échantillon. La lentille $L_1$ capture ces photons ainsi que les photons du faisceau principal qui ont subis une diffusion Rayleigh. La lentille $L_2$ concentre les photons collectés vers une fente de largeur $\delta x$, où quelques photons ambiants peuvent à ce point entrer dans le spectromètre double. Celui-ci se charge de filtrer les photons de la raie principale et sélectionne les photons de longueur d'onde $\lambda$. Comme le signal est faible, on utilise un photo-multiplicateur GaAs refroidit à $-20^oC$ pour amplifier le signal qui arrive sous forme d'impulsion électrique dans le compteur de photon. 
\begin{figure}[H]
	\centering
	\resizebox{!}{7cm}{
	\begin{tikzpicture}
		\node (O) at (0, 0) {};
		\node[above right=1cm and 1cm of O, inner sep=0pt] (Laser) {};
		\node[ above right=2cm and 2cm of Laser, inner sep=0pt] (Echantillon) {};
		\node[below right=0.5cm and .5cm of Echantillon, inner sep=0pt] (LaserOut) {};
		\node[below=1cm of Echantillon, inner sep=0pt] (Lentille1) {};
		\node[below=2cm of Lentille1, inner sep=0pt] (Lentille2){};
		\node[below=1cm of Lentille2, inner sep=0pt] (Fente) {};
		\node[below=1.5cm of Fente, inner sep=0pt] (SBottom) {};
		\node[below=.5cm of SBottom, inner sep=0pt] (PM) {};
		\node[below left=.5cm and 1cm of PM, inner sep=0pt] (PMBottom) {};
		\node[left=1cm of PMBottom, inner sep=0pt] (CPS) {};
		\node[right=.75 of Lentille1] {$L_1$};
		\node[right=.75 of Lentille2] {$L_2$};
		
		\draw[rotate around={45:(O)}] (O)++(-1.5cm, -.5cm) rectangle ++(3.17cm, 1cm) node[pos=.5, rotate around={45:(O)}] {Laser He-Ne};
		\draw (Echantillon)++(-1.5, 1) rectangle ++(2, -1) node[pos=.5] {Échantillon};
		\draw[<->] (Lentille1)++(-.5, 0) -- ++(1, 0);
		\draw[<->] (Lentille2)++(-.5, 0) -- ++(1, 0);
		\draw (Fente)++(2cm, 0) rectangle ++(-3cm, -1.5cm) node[pos=.5, text width=3cm, align=center] {Spectromètre \\ double};
		\draw (PM)++(2cm, 0) rectangle ++(-3cm, -1cm) node[pos=.5, text width=3cm, align=center] {Photo- \\ multiplicateur};
		\draw (CPS)++(0, 0.5cm) rectangle ++(-2cm, -1cm) node[pos=.5, text width=3cm, align=center] {Compteur \\ de photon};
		
		\draw (Laser) -- (Echantillon) -- (LaserOut);
		\draw (Echantillon) -- ++(-.5, -1.075) -- ++(0, -2) -- (Fente);
		\draw (Echantillon) -- ++(.5, -1.075) -- ++(0, -2) -- (Fente);
		\draw (SBottom)++(0, .06) -- (PM)++(0, -.11);
		\draw[->] (PMBottom) -- (CPS);
	\end{tikzpicture}
	}	
	\caption{Montage de spectroscopie Raman}
	\label{fig:montage}
\end{figure}

\subsection{Relation de dispersion}


\section{Résultats et discussion}\label{sec:resultats} % 25 points


\section{Conclusion}\label{sec:conclusion} % 10 points

\section{Extra} % 15 points

\printbibliography
%\bibliographystyle{abbvr} 
%\bibliography{raman_scattering_bib}
%\begin{thebibliography}{1}
%\bibitem{ref1} texte de la référence
%\end{thebibliography}

\end{document}
